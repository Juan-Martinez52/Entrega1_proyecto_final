\documentclass{article}
\usepackage[utf8]{inputenc}
\usepackage[spanish]{babel}
\usepackage{listings}
\usepackage{graphicx}
\graphicspath{ {images/} }
\usepackage{cite}

\renewcommand{\familydefault}{\sfdefault}

\begin{document}

\begin{titlepage}
    \begin{center}
        \vspace*{1cm}
            
        \Huge
        \textbf{Proyecto Final}
            
        \vspace{0.5cm}
        \LARGE
        
            
        \vspace{5cm}
            
        \textbf{Esteban Felipe Guiza Piñeros}\\
        \textbf{Juan David Martinez Bonilla}
            
        \vfill
            
        \vspace{0.8cm}
            
        \Large
        Despartamento de Ingeniería Electrónica y Telecomunicaciones\\
        Universidad de Antioquia\\
        Medellín\\
        Marzo de 2021
            
    \end{center}
\end{titlepage}
\begin{center}
\Huge
\textbf{Planteamiento de ideas }
\end{center}

\section{Idea 1}
\subsection{Planteamiento}
Se pensó un juego  con  un ambiente   similar al de un laberinto, donde el usuario tendra que moverce en direccion horizontal mientras el escenario avanzara por si mismo,  el jugador tendra que ir cumpliendo los objetivos y  tendra que enfrentarce a diversos obstaculos a lo largo del camino.

\subsection{Funcionamiento}
El juego consistira  en bloques que formen estructuras simples  donde  habran obstaculos moviles, trampas  y diversos objetos que el usuario puede recolectar  para potenciar las capacidades de su personaje y asi cumplir  el objetivo principal.El personaje contara  unicamente con  tres vidas para  terminar el juego.

\subsection{Objetivos}
 Lograr que el diseño del juego sea eficiente,  intentando que el usuario pueda avanzar de manera secuencial y cumpla la meta preprogramada. 

\section{Idea 2}
\subsection{Planteamiento}
Se pensó un juego donde el usuario será una nave espacial, esta tendrá que ir avanzando a traves de obstaculos que irán saliendo en el mapa, ademas a medida que el mapa avanza irán apareciendo diferentes potenciadores como escudos, velocidad esxtra o inmortalidad.

\subsection{Funcionamiento}
La idea principal será que la nave esquive diferentes objetos que irán saliendo de diferentes direcciones al frente de ella, los cuales irán variando en tamaño, velocidad y cantidad dependiendo de cuanto haya avanzado el jugador, cada cierto tiempo en el mapa apareceran potenciadores temporales que al pasar por encima permitiran saltar, esquivar o destruir los diferentes obstaculos al usuario.

\subsubsection{Objetivos}
El objetivo principal es durar la mayor cantidad de tiempo "vivo" logrando superar todos los obstaculos que progrsivamente van subiendo en complejidad






\end{document}

